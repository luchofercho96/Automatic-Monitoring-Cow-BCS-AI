\section{Índice condición corporal en bovinos}
Uno de los principales factores que debe tomar en cuenta el productor ganadero, es controlar la nutrición en sus animales, el método más conocido para identificar el nivel de nutrición en bovinos es la monitoreo de la condición corporal, consiste en evaluar mediante una apreciación visual  las reservas corporales de grasa y músculo, bajo un patrón preestablecido al que se le ha dado valores numéricos arbitrarios, estos valores están ordenados conforme a una escala que en las razas británicas y  continentales va de 1 a 5 y en las índicas y sus cruzas de 1 a 9 \cite{ImportanciaCria}. 

Las investigaciones indican que existe un fuerte vínculo entre la condición corporal de una vaca y su rendimiento reproductivo, está relacionada con muchos aspectos críticos de la producción por ejemplo ayuda a los productores a tomar decisiones de gestión con respecto a la calidad y cantidad de alimento necesarios para garantizar la salud, optimizar la producción y reproducción \cite{Eversole2014BodyCows,Arias2008FactoresLeche}.

\vspace{5mm} %5mm vertical space
\subsection{Relación entre condición corporal y eficiencia reproductiva}

El porcentaje de eficiencia reproductiva está directamente relacionado con dos factores: el número de parto y la condición corporal.  (Tabla 1). Se considera que vacas con una condición corporal mayor a 3, tienen un 29\% mayores tasas de preñez  comparada con vacas con una condición corporal menor a 2.5 \cite{RelacionHolstein}. La pobre condición corporal afecta no solo la tasa de preñez sino el intervalo entre partos (IEP) y la edad del ternero al destete \cite{RelacionHolstein}.

\begin{table}[h]
\caption{Relación del parto y la condición corporal con
tasas de preñez (\%)}
\label{tab:tabla1}
\begin{adjustbox}{max width=\textwidth}
\begin{tabular}{| c | c | c | c | }
\hline
\multicolumn{1}{ |c| }{Parto} & \multicolumn{3}{ |c| }{Condición corporal} \\ \hline
N  & $<2.0$ & $2.5$ & $>3.0$ \\ \hline
1 & 20 & 53 & 90 \\
2 & 28 & 50 & 84 \\
3 & 23 & 60 & 90 \\
4-7 & 48 & 72 & 92 \\
$> 8$ & 37 & 67 & 89 \\ \hline
\end{tabular}
\end{adjustbox}
\end{table}

Sin embargo, el método actual para medir BCS en granjas lecheras modernas todavía es manual. El BCS manual lleva mucho tiempo en granjas grandes y requiere mano de obra capacitada \cite{Halachmi2013AutomaticImaging}. Otro problema asociado con el puntaje manual es la subjetividad del proceso, el puntaje depende de la persona que realiza la medición.a pesar de lo anteriormente mencionado  menos del 5\% de las granjas lecheras de los EEUU han adoptado esta práctica en la cadena de producción \cite{Azzaro2011ObjectiveImages}.

Las principales razones que desalientan el uso de las técnicas de evaluación de BCS son la falta de informes computarizados, la  subjetividad y el tiempo de capacitación de técnicos en la granja. Et Al \cite{Schroder2006InvitedThickness} sugieren que un BCS automático requiere menos tiempo, es menos estresante para el animal, más objetivo y consistente, y posiblemente sea más rentable. Por lo tanto, el desarrollo de un dispositivo automático basado en imágenes digitales para monitorear BCS es de interés económico.

\vspace{5mm} %5mm vertical space
\subsection{Relación entre condición corporal y eficiencia reproductiva}

El porcentaje de eficiencia reproductiva está directamente relacionado con dos factores: el número de parto y la condición corporal.  (Tabla 1). Se considera que vacas con una condición corporal mayor a 3, tienen un 29\% mayores tasas de preñez  comparada con vacas con una condición corporal menor a 2.5 \cite{RelacionHolstein}. La pobre condición corporal afecta no solo la tasa de preñez sino el intervalo entre partos (IEP) y la edad del ternero al destete \cite{RelacionHolstein}.

\vspace{5mm} %5mm vertical space